%%%%%%%%%%%%%%%%%%%%%%%%%%%%%%%%%%%%%%%%%%%%%%%%%%%%%%%%%%%%%%%%%%%%%%%%%%%%%%%%
%                                                                              %
% BlackArch Linux Guide                                                        %
%                                                                              %
%%%%%%%%%%%%%%%%%%%%%%%%%%%%%%%%%%%%%%%%%%%%%%%%%%%%%%%%%%%%%%%%%%%%%%%%%%%%%%%%


%%% BEGIN %%%
\documentclass[a4paper, twoside, 11pt]{article}


%%% INCLUDES %%%
\renewcommand{\familydefault}{\sfdefault}
\usepackage{verbatim}
\usepackage[utf8]{inputenc}
\usepackage{geometry}
\usepackage{array}
\usepackage{graphicx}
\usepackage{supertabular}
\usepackage{verbatim}
\usepackage{pstricks}
\usepackage{fancyhdr}
\usepackage{fancyvrb}
\usepackage{tocloft}
\usepackage{color}
\usepackage{enumerate}
\usepackage[%
pdftitle={BlackArch Linux, The BlackArch Linux Guide},
pdfsubject={BlackArch Linux, The BlackArch Linux Guide},
pdfauthor={BlackArch Linux, BlackArch Linux},
pdfkeywords={BlackArch Linux, Penetration Testing, Security, Hacking, Linux},
pdfcenterwindow=true
]{hyperref}


%%% LAYOUT %%%
\setlength{\parindent}{0em}
\setlength{\parskip}{1.5ex plus0.5ex minus0.5ex}
\geometry{left=2.5cm, textwidth=16cm, top=3.0cm, textheight=25cm, bottom=3cm}
\widowpenalty=2000
\clubpenalty=1000
\frenchspacing
\sloppy
\pagecolor{black}
\color{gray}
\hypersetup{pdftex=true, colorlinks=true, breaklinks=true, linkcolor=red,
menucolor=red, pagecolor=red, urlcolor=red}
\setcounter{tocdepth}{10}
\setcounter{secnumdepth}{10}


%%% HEADER / FOOTER %%%
\setlength{\headheight}{1.0cm}
\setlength{\headsep}{1.0cm}
\lhead{{\includegraphics[width=1cm,height=1cm]{logo.png}}}
\rhead{The BlackArch Linux Guide}
%\lfoot{\small{foo}}
%\rfoot{\small{bar}}


%%% OWN MACROS %%%
% put own macros here, like renewcommand newcommand etc.


%%% DOCUMENT %%%
\begin{document}

\pagestyle{empty}
\begin{center}
\begin{figure}[htbp]
\centering
\vspace{1cm}
\Huge{\textbf{BlackArch \color{red}Linux}}\\
\vspace{2cm}
\includegraphics[width=8cm]{logo.png}
\label{fig:logo}
\end{figure}
\vspace{1cm}
\Huge{\textbf{The BlackArch Linux Guide}}\\
\vspace{1cm}
\Large{\color{red}http://www.blackarch.org/}\\
\vspace{0.5cm}
\LARGE{\today}
\end{center}
\newpage
\tableofcontents
\newpage
\pagestyle{fancy}

\section{Introduction}

\subsection{What is BlackArch Linux?}
\href{http://www.blackarch.org}{BlackArch Linux} is a lightweight expansion to
Arch Linux for penetration testers.
\\\\
The toolset is distributed as an Arch Linux
\href{https://wiki.archlinux.org/index.php/Unofficial\_User\_Repositories}
{unofficial user repository} so you can install BlackArchLinux on top of an
existing Arch Linux installation. Packages may be installed individually or by
category.
\\\\
We currently have over 650 tools in our toolset and the repository is constantly
expanding. All tools are thoroughly tested before being added to the codebase to
maintain the quality of the repository.

\subsection{The story of BlackArch Linux}
foo bar

\subsection{Supported platforms}
foo bar

\subsection{Get involved}
You can get in touch with the BlackArch team. Just check out the following:
\\\\
Web: \url{http://www.blackarch.org/}
\\\\
Mail: \href{mailto:blackarchlinux@gmail.com}{blackarchlinux@gmail.com}
\\\\
IRC: \url{irc://irc.freenode.net/blackarch}

%%%%%%%%%%%%%%%%%%%%%%%%%%%%%%%%%%%%%%%%%%%%%%%%%%%%%%%%%%%%%%%%%%%%%%%%%%%%%%%%

\section{User Guide}

\subsection{Installation}
The following sections will show you how to setup the BlackArch repository and
install packages. BlackArch supports both, installing from repository using
binary packages as well as compiling and installing from sources.
\\\\
BlackArch is compatible with normal Arch installations. It acts as an unofficial
user repository. If you want an ISO instead, see the
\href{http://www.blackarch.org/download.html#iso}{Live ISO} section.
\\\\

\subsubsection{Setting up repository}
There are 6 steps in order to setup and use the BlackArch repository
successfully. You must follow the instuctions in order. Do not add
\textbf{blackarch} to your \textit{pacman.conf} file without following steps 0
to 2 first.
\begin{enumerate}
\item If you have installed BlackArch before and there is an existing
\textbf{[blackarch]} entry in \textit{/etc/pacman.conf}, remove or comment out
the entry and run \textit{pacman -Syy}.
\item Run the following as root. This is for package signing.
{\small
\begin{verbatim}
wget -q http://blackarch.org/keyring/blackarch-keyring.pkg.tar.xz{,.sig}
gpg --keyserver hkp://pgp.mit.edu --recv 4345771566D76038C7FEB43863EC0ADBEA87E4E3
gpg --keyserver-o no-auto-key-retrieve --with-f blackarch-keyring.pkg.tar.xz.sig
pacman-key --init
rm blackarch-keyring.pkg.tar.xz.sig
pacman --noc -U blackarch-keyring.pkg.tar.xz
\end{verbatim}
}
\item If possible, please verify the signing key's fingerprint against as many
sources as possible.
\item Append the following lines to your /etc/pacman.conf file:
{\small
\begin{verbatim}
[blackarch]
Server = <mirror_site>/$repo/os/$arch
\end{verbatim}
}
Replace \textit{\textless mirror\_site\textgreater} with a mirror site of your
choosing. Please use one of our official mirrors.
\item Now run:
{\small
\begin{verbatim}
$ sudo pacman -Syyu
\end{verbatim}
}
\end{enumerate}

\subsubsection{Installing packages}
You may now install tools from the blackarch repository.

\begin{enumerate}
\item To list all of the available tools, run
{\small
\begin{verbatim}
$ sudo pacman -Sgg | grep blackarch | cut -d' ' -f2 | sort -u
\end{verbatim}
}
\item To install all of the tools, run
{\small
\begin{verbatim}
$ sudo pacman -S blackarch
\end{verbatim}
}
\item To install a category of tools, run
{\small
\begin{verbatim}
$ sudo pacman -S blackarch-<category>
\end{verbatim}
}
\item To see the blackarch categories, run
{\small
\begin{verbatim}
$ sudo pacman -Sg | grep blackarch
\end{verbatim}
}
\end{enumerate}

\subsubsection{Installing packages from source}
As part of an alternative method of installation, you can build the BlackArch
packages from source. You can find the PKGBUILDs on
\href{https://github.com/BlackArch/blackarch/tree/master/packages}{github}. To
build the entire repo, you can use the
\href{https://github.com/BlackArch/blackman}{blackman} tool.
\begin{itemize}
\item First, you have to install blackman. If the BlackArch package repository
is setup on your machine, you can install blackman:
{\small
\begin{verbatim}
pacman -S blackman
\end{verbatim}
}
\item You can build and install blackman from source:
{\small
\begin{verbatim}
mkdir blackman
cd blackman
wget https://raw2.github.com/BlackArch/blackarch/master/packages/blackman/PKGBUILD
# Make sure the PKGBUILD has not been maliciously tampered with.
makepkg -s
\end{verbatim}
}
\item Download, compile and install packages:
{\small
\begin{verbatim}
$ sudo blackman -i package
\end{verbatim}
}
\item Download, compile and install whole category:
{\small
\begin{verbatim}
$ sudo blackman -g group
\end{verbatim}
}
\item Download, compile and install all of the BlackArch tools:
{\small
\begin{verbatim}
$ sudo blackman -a
\end{verbatim}
}
\item To list the blackarch categories:
{\small
\begin{verbatim}
$ blackman -l
\end{verbatim}
}
\item To list category tools:
{\small
\begin{verbatim}
$ blackman -p category
\end{verbatim}
}
\end{itemize}

\subsubsection{Installing from live-, netinstall- ISO or ArchLinux}
You can install BlackArch Linux from one of our live- or netinstall-ISOs.\\See
\url{http://www.blackarch.org/download.html#iso}. The following steps are
required after the ISO boot up.
\begin{itemize}
\item Install blackarch-install-scripts package:
{\small
\begin{verbatim}
$ sudo pacman -S blackarch-install-scripts
\end{verbatim}
}
\item Run
{\small
\begin{verbatim}
$ sudo blackarch-install
\end{verbatim}
}
\end{itemize}

%%%%%%%%%%%%%%%%%%%%%%%%%%%%%%%%%%%%%%%%%%%%%%%%%%%%%%%%%%%%%%%%%%%%%%%%%%%%%%%%

\section{Developer Guide}

\subsection{Repository structure}
foo bar

\subsubsection{broken}
foo bar

\subsubsection{docs}
foo bar

\subsubsection{mirror}
foo bar

\subsubsection{misc}
foo bar

\subsubsection{packages}
foo bar

\subsubsection{scripts}
foo bar

\subsection{Contributing to repository}
This section shows you how to contribute to the BlackArch Linux project. We
accept pull requests of all sizes, from tiny typo fixes to new packages.\\For
help, suggestions, or questions feel free to contact us.
\\\\
Everyone is welcome to contribute. All contributions are appreciated.

\subsubsection{Required tutorials}
Please read the following tutorials before contributing:
\begin{itemize}
\item
\href{https://wiki.archlinux.org/index.php/Arch\_Packaging\_Standards)}{Arch
Packaging Standards}
\item \href{https://wiki.archlinux.org/index.php/Creating\_Packages}{Creating
Packages}
\item \href{https://wiki.archlinux.org/index.php/PKGBUILD}{PKGBUILD}
\item \href{https://wiki.archlinux.org/index.php/Makepkg}{Makepkg}
\end{itemize}

\subsubsection{Steps for contributing}
In order to submit your changes to the BlackArchLinux project, follow these
steps:
\begin{enumerate}
\item Fork the repository from
\url{https://github.com/BlackArchLinux/blackarchlinux}
\item Hack the necessary files, (e.g. PKGBUILD, .patch files, etc).
\item Commit your changes.
\item Push your changes.
\item Ask us to merge in your changes, preferably through a pull request.
\end{enumerate}

\subsubsection{Example}
The following example demonstrates submitting a new package to the BlackArch
project. We use \href{https://wiki.archlinux.org/index.php/yaourt}{yaourt}
(you can use pacaur as well) to fetch a pre-existing PKGBUILD file for
\textbf{nfsshell} from the \href{https://aur.archlinux.org/}{AUR} and adjust it
according to our needs.

\paragraph{Fetch PKGBUILD}
Fetch the \textit{PKGBUILD} file using yaourt or pacaur:
{\small
\begin{verbatim}
user@blackarchlinux $ yaourt -G nfsshell
==> Download nfsshell sources
x LICENSE
x PKGBUILD
x gcc.patch
user@blackarchlinux $ cd nfsshell/
\end{verbatim}
}

\paragraph{Clean up PKGBUILD}
Clean up the \textit{PKGBUILD} file and save some time:
{\small
\begin{verbatim}
user@blackarchlinux nfsshell $ ./blarckarch/scripts/prep PKGBUILD
cleaning 'PKGBUILD'...
expanding tabs...
removing vim modeline...
removing id comment...
removing contributor and maintainer comments...
squeezing extra blank lines...
removing '|| return'...
removing leading blank line...
removing $pkgname...
removing trailing whitespace...
\end{verbatim}
}

\paragraph{Adjust PKGBUILD}
Adjust the \textit{PKGBUILD} file:
{\small
\begin{verbatim}
user@blackarchlinux nfsshell $ vi PKGBUILD
\end{verbatim}
}

\paragraph{Build the package}
Build the package:
{\small
\begin{verbatim}
user@blackarchlinux nfsshell $ makepkg -sf
==> Making package: nfsshell 19980519-1 (Mon Dec  2 17:23:51 CET 2013)
==> Checking runtime dependencies...
==> Checking buildtime dependencies...
==> Retrieving sources...
-> Downloading nfsshell.tar.gz...
% Total    % Received % Xferd  Average Speed   Time    Time     Time
CurrentDload  Upload   Total   Spent    Left  Speed100 29213  100 29213    0
0  48150      0 --:--:-- --:--:-- --:--:-- 48206
-> Found gcc.patch
-> Found LICENSE
...
<lots of build process and compiler output here>
...
==> Leaving fakeroot environment.
==> Finished making: nfsshell 19980519-1 (Mon Dec  2 17:23:53 CET 2013)
\end{verbatim}
}

\paragraph{Install and test the package}
Install and test the package:
{\small
\begin{verbatim}
user@blackarchlinux nfsshell $ pacman -U nfsshell-19980519-1-x86_64.pkg.tar.xz
user@blackarchlinux nfsshell $ nfsshell # test it
\end{verbatim}
}

\paragraph{Add, commit and push package}
Add, commit and push the package
{\small
\begin{verbatim}
user@blackarchlinux nfsshell $ cd /blackarchlinux/packages
user@blackarchlinux ~/blackarchlinux/packages $ mv ~/nfsshell .
user@blackarchlinux ~/blackarchlinux/packages $ git add nfsshell && git commit
nfsshell && git push
\end{verbatim}
}

\paragraph{Create a pull request}
Create a pull request on \href{https://github.com/}{github.com}
{\small
\begin{verbatim}
firefox https://github.com/<contributor>/blackarchlinux
\end{verbatim}
}

\subsubsection{Requests}
\begin{enumerate}
\item Don't add \textbf{Maintainer} or \textbf{Contributor} comments to
\textit{PKGBUILD} files. Add maintainer and contributor names to the
AUTHORS section of BlackArch guide.
\item For the sake of consistency, please follow the general style of the other
\textit{PKGBUILD} files in the repo and use two-space indentation.
\end{enumerate}

\subsubsection{General tips}
\href{http://wiki.archlinux.org/index.php/Namcap}{namcap} can check packages for
errors.

%%% APPENDIX %%%
\appendix
%%%%%%%%%%%%%%%%%%%%%%%%%%%%%%%%%%%%%%%%%%%%%%%%%%%%%%%%%%%%%%%%%%%%%%%%%%%%%%%%
%                                                                              %
% BlackArch Linux appendix template                                            %
%                                                                              %
%%%%%%%%%%%%%%%%%%%%%%%%%%%%%%%%%%%%%%%%%%%%%%%%%%%%%%%%%%%%%%%%%%%%%%%%%%%%%%%%

\appendix

\chapter{Appendix}

\section{FAQs}

\section{AUTHORS}
The following people have contributed directly to BlackArch:
\begin{itemize}
\item Evan Teitelman <teitelmanevan at gmail dot com>
\item Tyler Bennnett <tylerb at trix2voip dot com>
\item Levon Kayan <noptrix at nullsecurity dot net>
\item Jeremy Lynch <jl at adminempire dot com>
\item Ari Mizrahi <codemunchies at gmail dot com>
\item fnord0 <fnord0 at riseup dot net>
\item nrz <nrz at nullsecurity dot net>
\item Ellis Kenyo <elken.tdos at gmail dot com>
\item CaledoniaProject <the.warl0ck.1989 at gmail dot com>
\item sudokode <sudokode at gmail dot com>
\item Valentin Churavy <v.churavy at gmail dot com>
\item Boy Sandy Gladies Arriezona <reno.esper at gmail dot com>
\item Mathias Nyman <None>
\item Johannes Löthberg <demizide at gmail dot com>
\end{itemize}
The following people contributed directly to ArchPwn, which has been merged
into BlackArch:
\begin{itemize}
\item Francesco Piccinno <stack.box at gmail dot com>
\item jensp <jens at jenux.homelinux dot org>
\item Valentin Churavy <v.churavy at gmail dot com>
\end{itemize}
We have taken build code from the following people:
\begin{itemize}
\item  3ED <krzysztof1987 at gmail dot com>
\item  AUR Perl <aurperl at juster dot info>
\item  Aaron Griffin <aaron at archlinux dot org>
\item  Abakus <java5 at arcor dot de>
\item  Adam Wolk <netprobe at gmail dot com>
\item  Aleix Pol <aleixpol at kde dot org>
\item  Aleshus <aleshusi at gmail dot com>
\item  Alessandro Pazzaglia <jackdroido at gmail dot com>
\item  Alessandro Sagratini <ale\_sagra at hotmail dot com>
\item  Alex Cartwright <alexc223 at googlemail dot com>
\item  Alexander De Sousa <archaur.xandy21 at spamgourmet dot com>
\item  Alexander Rødseth <rodseth at gmail dot com>
\item  Allan McRae <allan at archlinux dot org>
\item  AmaN <gabroo.punjab.da at gmail dot com>
\item  Andre Klitzing <aklitzing at online dot de>
\item  Andrea Scarpino <andrea at archlinux dot org>
\item  Andreas Schönfelder <passtschu at freenet dot de>
\item  Andrej Gelenberg <andrej.gelenberg at udo dot edu>
\item  Angel Velasquez <angvp at archlinux dot org>
\item  Antoine Lubineau <antoine at lubignon dot info>
\item  Anton Bazhenov <anton.bazhenov at gmail>
\item  Arkham <arkham at archlinux dot us>
\item  Arthur Danskin <arthurdanskin at gmail dot com>
\item  Balda Baldanos \_at\_ gmail dot com
\item  Balló György <ballogyor+arch at gmail dot com>
\item  Bartek Piotrowski <barthalion at gmail dot com>
\item  Bartosz Feński <fenio at debian dot org>
\item  Bartłomiej Piotrowski <nospam at bpiotrowski dot pl>
\item  Bogdan Szczurek <thebodzio at gmail dot com>
\item  Brad Fanella <bradfanella at archlinux dot us>
\item  Brian Bidulock <bidulock at openss7 dot org>
\item  C Anthony Risinger <anthony at xtfx dot me>
\item  CRT <crt.011 at gmail dot com>
\item  Can Celasun <dcelasun at gmail dot com>
\item  Chaniyth <chaniyth at yahoo dot com>
\item  Chris Brannon <cmbrannon79 at gmail dot com>
\item  Chris Giles <Chris.G.27 at gmail dot com> \& daschu117
\item  Christoph Siegenthaler <csi at gmx dot ch>
\item  Christoph Zeiler <archNOSPAM at moonblade dot org>
\item  Clément DEMOULINS <clement at archivel dot fr>
\item  Corrado Primier <bardo at aur.archlinux dot org>
\item  Daenyth <Daenyth+Arch at gmail dot com>
\item  Dale Blount <dale at archlinux dot org>
\item  Damir Perisa <damir.perisa at bluewin dot ch>
\item  Dan Fuhry <dan at fuhry dot us>
\item  Dan Serban <dserban01 at yahoo dot com>
\item  Daniel A. Campoverde Carrión
\item  Daniel Golle
\item  Daniel Griffiths <ghost1227 at archlinux dot us>
\item  Daniel J Griffiths <ghost1227 at archlinux dot us>
\item  Daniel Micay <danielmicay at gmail dot com>
\item  Dave Reisner <dreisner at archlinux dot org>
\item  Dawid Wrobel <cromo at klej dot net>
\item  Devaev Maxim <mdevaev at gmail dot com>
\item  Devin Cofer <ranguvar at archlinux dot us>
\item  DigitalPathogen <aur at InfoSecResearchLabs dot co dot uk>
\item  DigitalPathogen <aur at digitalpathogen dot co dot uk>
\item  Dmitry A. Ilyashevich <dmitry.ilyashevich at gmail dot com>
\item  Dominik Heidler <dheidler at gmail dot com>
\item  DrZaius <lou at fakeoutdoorsman dot com>
\item  Ebubekir KARUL <ebubekirkarul at yandex dot com>
\item  Eduard "bekks" Warkentin <eduard.warkentin at gmail dot com>
\item  Elmo Todurov <todurov at gmail dot com>
\item  Emmanuel Gil Peyrot <linkmauve at linkmauve dot fr>
\item  Eric Belanger <eric at archlinux dot org>
\item  Ermak <ermak at email dot it>
\item  Evangelos Foutras <evangelos at foutrelis dot com>
\item  Fabian Melters <melters at gmail dot com>
\item  Fabiano Furtado <fusca14 at gmail dot com>
\item  Federico Quagliata (quaqo) <linux at quaqo dot org>
\item  Firmicus <francois.archlinux dot org>
\item  Florian Pritz <bluewind at jabber dot ccc dot de>
\item  Florian Pritz <flo at xinu dot at>
\item  Francesco Piccinno <stack.box at gmail dot com>
\item  François Charette <francois at archlinux dor org>
\item  Gaetan Bisson <bisson at archlinux dot org>
\item  Geoffroy Carrier <geoffroy.carrier at koon.fr>
\item  Georg Grabler (STiAT) <ggrabler at gmail dot com>
\item  George Hilliard <gh403 at msstate dot edu>
\item  Gerardo Exequiel Pozzi <vmlinuz386 at yahoo dot com dot ar>
\item  Gilles CHAUVIN <gcnweb at gmail dot com>
\item  Giovanni Scafora <giovanni at archlinux dot org>
\item  Gordin <9ordin  at t gmail dot com>
\item  Guillaume ALAUX <guillaume at archlinux dot org>
\item  Guillermo Ramos <0xwille at gmail dot com>
\item  Gustavo Alvarez <sl1pkn07 at gmail dot com>
\item  Hugo Doria <hugo at archlinux dot org>
\item  Hyacinthe Cartiaux <hyacinthe.cartiaux at free dot fr>
\item  James Fryman <jfryman at gmail dot com>
\item  Jan "heftig" Steffens <jan.steffens at gmail dot com>
\item  Jan de Groot <jgc at archlinux dot org>
\item  Jaroslav Lichtblau <dragonlord at aur dot archlinux dot org>
\item  Jaroslaw Swierczynski <swiergot at aur dot archlinux dot org>
\item  Jason Chu <jason at archlinux dot org>
\item  Jason R Begley (jayray at digitalgoat dot com>
\item  Jason Rodriguez <jason\item aur at catloaf dot net>
\item  Jason St. John <jstjohn at purdue dot edu>
\item  Jawmare <victor2008 at gmail dot com>
\item  Jeff Mickey <jeff at archlinux dot org>
\item  Jens Pranaitis <jens at chaox dot net>
\item  Jens Pranaitis <jens at jenux dot homelinux dot org>
\item  Jinx <jinxware at gmail dot com>
\item  John D Jones III <j[nospace]n[nospace]b[nospace]e[nospace]k[nospace]1972 at gmail dot com>
\item  John Proctor <jproctor at prium dot net>
\item  Jon Bergli Heier <snakebite at jvnv dot net>
\item  Jonas Heinrich <onny at project\item insanity dot org>
\item  Jonathan Steel <jsteel at aur dot archlinux dot org>
\item  Joris Steyn <jorissteyn at gmail dot com>
\item  Josh VanderLinden <arch at cloudlery dot com>
\item  Jozef Riha <jose1711 at gmail dot com>
\item  Judd Vinet <jvinet at zeroflux dot org>
\item  Juergen Hoetzel <jason at archlinux dot org>
\item  Juergen Hoetzel <juergen at archlinux dot org>
\item  Justin Davis <jrcd83 at gmail dot com>
\item  Kaiting Chen <kaitocracy at gmail dot com>
\item  Kaos < gianlucaatlas at gmail dot com >
\item  Kevin Piche <kevin at archlinux dot org>
\item  Kory Woods <kory \_at\_ virlo >dot< net>
\item  Kyle Keen <keenerd at gmail dot com>
\item  Larry Hajali <larryhaja at gmail dot com>
\item  LeCrayonVert < greenarrow at archlinux dot us>
\item  Le\_suisse <lesuisse dot dev+aur at gmail dot com>
\item  Lekensteyn <lekensteyn at gmail dot com>
\item  Limao Luo <luolimao+AUR at gmail dot com>
\item  Lucien Immink <l.immink at student dot fnt dot hvu dot nl>
\item  Lukas Fleischer <archlinux at cryptocrack dot de>
\item  Manolis Tzanidakis
\item  Marcin "avalan" Falkiewicz <avalatron at gmail dot com>
\item  Mariano Verdu <verdumariano at gmail dot com>
\item  Marti Raudsepp <marti at juffo dot org>
\item  MatToufoutu <mattoufootu at gmail dot com>
\item  Matthew Sharpe <matt.sharpe at gmail dot com>
\item  Mauro Andreolini <mauro.andreolini at unimore dot it>
\item  Max Pray a.k.a. Synthead <synthead at gmail dot com>
\item  Max Roder <maxroder at web dot de>
\item  Maxwell Pray a.k.a. Synthead <synthead at gmail dot com>
\item  Maxwell Pray a.k.a. Synthead <synthead1 at gmail dot com>
\item  Mech <tiago.bmp at gmail dot com>
\item  Michael Düll <mail at akurei dot me>
\item  Michael P <ptchinster at archlinux dot us>
\item  Michal Krenek <mikos at sg1 dot cz>
\item  Michal Zalewski <lcamtuf at coredumpdotcx>
\item  Miguel Paolino <mpaolino at gmail dot com>
\item  Miguel Revilla <yo at miguelrevilla dot com>
\item  Mike Roberts <noodlesgc at gmail dot com>
\item  Mike Sampson <mike at sambodata dot com>
\item  Nassim Kacha <nassim.kacha at gmail dot com>
\item  Nicolas Pouillard <nicolas.pouillard at gmail dot com>
\item  Nicolas Pouillard https://nicolaspouillard.fr
\item  Niklas Schmuecker (IRC: nisc) <nschmuecker gmail dot com>
\item  Oleander Reis <oleander at oleander dot cc>
\item  Olivier Le Moal <mail at olivierlemoal dot fr>
\item  Olivier Médoc "oliv" <o\_medoc at yahoo dot fr>
\item  Pascal E. <archlinux at hardfalcon dot net>
\item  Patrick Leslie Polzer <leslie.polzer at gmx dot net>
\item  Paul Mattal <paul at archlinux dot org>
\item  Paul Mattal <pjmattal at elys dot com>
\item  Pengyu CHEN <cpy.prefers.you at gmail dot com>
\item  Peter Wu <lekensteyn at gmail dot com>
\item  Philipp 'TamCore' B. <philipp at tamcore dot eu>
\item  Pierre Schmitz <pierre at archlinux dot de>
\item  Pranay Kanwar <pranay.kanwar at gmail dot com>
\item  Pranay Kanwar <warl0ck at metaeye dot org>
\item  PyroPeter <googlemail dot com at abi1789>
\item  PyroPeter <googlemail.com at abi1789>
\item  Ray Rashif <schiv at archlinux dot org>
\item  Remi Gacogne <rgacogne\item arch at coredump dot fr>
\item  Renan Fernandes <renan at kauamanga dot com dot br>
\item  Richard Murri <admin at richardmurri dot com>
\item  Roberto Alsina <ralsina at kde dot org>
\item  Robson Peixoto <robsonpeixoto at gmail dot com>
\item  Roel Blaauwgeers <roel at ttys0 dot nl>
\item  Rorschach <r0rschach at lavabit dot com>
\item  Ruben Schuller <shiml at orgizm dot net>
\item  Rudy Matela <rudy at matela dot com dot br>
\item  Ryon Sherman <ryon.sherman at gmail dot com>
\item  Sabart Otto \item  Seberm <seberm at gmail dot com>
\item  SakalisC <chrissakalis at gmail dot com>
\item  Sam Stuewe <halosghost at archlinux dot info>
\item  SanskritFritz <SanskritFritz at gmail dot com)
\item  Sarah Hay <sarahhay at mb dot sympatico dot ca>
\item  Sebastian Benvenuti <sebastianbenvenuti at gmail dot com>
\item  Sebastian Nowicki <sebnow at gmail dot com>
\item  Sebastien Duquette <ekse.0x at gmail dot com>
\item  Sebastien LEDUC <sebastien at sleduc dot fr>
\item  Sebastien Leduc <sebastien at sleduc dot fr>
\item  Sergej Pupykin <pupykin.s+arch at gmail dot com>
\item  Sergio Rubio <rubiojr at biondofu dot net>
\item  Sheng Yu <magicfish1990 at gmail dot com>
\item  Simon Busch <morphis at gravedo dot de>
\item  Simon Legner <Simon.Legner at gmail dot com>
\item  Sirat18 <aur at sirat18 dot de>
\item  SpepS <dreamspepser at yahoo dot it>
\item  Spider.007 <archPackage at spider007 dot net>
\item  Stefan Seering
\item  Stephane Travostino <stephane.travostino at gmail dot com>
\item  Stéphane Gaudreault <stephane at archlinux dot org>
\item  Sven Kauber <celeon at gmail dot com>
\item  Sven Schulz <omee at archlinux dot de>
\item  Sébastien Duquette <ekse.0x at gmail dot com>
\item  Sébastien Luttringer <seblu at archlinux dot org>
\item  TDY <tdy at gmx dot com>
\item  Teemu Rytilahti <tpr at iki dot fi>
\item  Testuser\_01 <mail at nico\item siebler dot de>
\item  Thanx <thanxm at gmail dot com>
\item  Thayer Williams <thayer at archlinux dot org>
\item  Thomas S Hatch <thatch45 at gmail dot com>
\item  Thorsten Töpper <atsutane\item aur at freethoughts dot de>
\item  Thorsten Töpper <atsutane\item tu at freethoughts dot de>
\item  Tilmann Becker <tilmann.becker at freenet dot de>
\item  Timothy Redaelli <timothy.redaelli at gmail dot com>
\item  Timothée Ravier <tim at siosm dot fr>
\item  Tino Reichardt
\item  Tobias Kieslich <tobias at justdreams dot de>
\item  Tobias Powalowski <tpowa at archlinux dot org>
\item  Tom K <tomk at runbox dot com>
\item  Tom Newsom <Jeepster at gmx dot co dot uk>
\item  Tomas Lindquist Olsen <tomas at famolsen dot dk>
\item  Travis Willard <travisw at wmpub dot ca>
\item  Valentin Churavy <v.churavy at gmail dot com>
\item  ViNS <gladiator at fastwebnet dot it>
\item  Vlatko Kosturjak <kost at linux dot hr>
\item  Wes Brown <wesbrown18 at gmail dot com>
\item  William Rea <sillywilly at gmail dot com>
\item  Xavier Devlamynck <magicrhesus at ouranos dot be>
\item  Xiao\item Long Chen <chenxiaolong at cxl dot epac dot to>
\item  aeolist <aeolist at hotmail dot com>
\item  ality at pbrane dot org
\item  astaroth <astaroth\_ at web dot de>
\item  bender02 at archlinux dot us
\item  billycongo <billycongo at gmail dot com>
\item  bslackr <brendan at vastactive dot com>
\item  cbreaker <cbreaker at tlen dot pl>
\item  chimeracoder <dev@chimeracoder.net>
\item  damir <damir at archlinux.org>
\item  danitool
\item  darkapex <me at jailuthra dot in>
\item  daronin
\item  dkaylor <dpkaylor at gmail dot com>
\item  dobo <dobo90\_at\_gmail dot com>
\item  dorphell <dorphell at archlinux dot org>
\item  evr <evanroman  at  gmail>
\item  fnord0 <fnord0 at riseup dot net>
\item  fxbru <frxbru at gmail>
\item  hcar
\item  icarus <icarus.roaming at gmail dot com>
\item  iceman <icemanf at gmail dot com>
\item  kastor <kastor at fobos dot org dot ar>
\item  kfgz <kfgz at interia dot pl>
\item  linuxSEAT <linuxSEAT at gmail dot com>
\item  m4xm4n <max at maxfierke dot com>
\item  mar77i <mysatyre at gmail dot com>
\item  marc0s <marc0s at fsfe dot org>
\item  mickael9 <mickael9 at gmail dot com>
\item  nblock <nblock at archlinux dot us>
\item  nofxx <x at nofxx dot com>
\item  onny <onny at project
\item  pootzko <pootzko at gmail dot com>
\item  revel <revel at muub dot net>
\item  rich\_o <rich\_o at lavabit dot com>
\item  s1gma <s1gma at mindslicer dot com>
\item  sandman <r.coded at gmail dot com>
\item  sebikul <sebikul at gmail dot com>
\item  sh0 <mee at sh0 dot org>
\item  shild <sxp at bk dot ru>
\item  simo <simo at archlinux dot org>
\item  snuo
\item  sudokode <sudokode at gmail dot com>
\item  tobias <tobias at archlinux dot org>
\item  trashstar <trash at ps3zone dot org>
\item  unexist <unexist at subforge dot org>
\item  untitled <rnd0x00 at gmail dot com>
\item  virtuemood <virtue at archlinux dot us>
\item  wido <widomaker2k7 at gmail dot com>
\item  wodim <neikokz at gmail dot com>
\item  yannsen <ynnsen at gmail dot com>
\end{itemize}


\end{document}

%%% EOF %%%
